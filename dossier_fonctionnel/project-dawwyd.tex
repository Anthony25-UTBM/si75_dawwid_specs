\documentclass[12pt]{article}

\usepackage[utf8]{inputenc}
\usepackage[T1]{fontenc}
\usepackage[francais]{babel}
\usepackage{multirow}
\usepackage{array}
\usepackage{color}
\usepackage{stmaryrd}
\usepackage{fancyhdr}
\usepackage{afterpage}
\usepackage{fullpage}
\usepackage{geometry}
\usepackage{setspace}
\usepackage{enumitem}
\usepackage{hyperref}
\usepackage{graphicx}

% Enlève les contours des liens
\hypersetup{
    linkbordercolor={1 1 1},
    citebordercolor={1 1 1},
    urlbordercolor={1 1 1},
    colorlinks=true,
    linkcolor=black,
    urlcolor=blue
}
\PassOptionsToPackage{hyphens}{url}\usepackage{hyperref}


\title{\textbf{SI75 -- Logiciel de Commande Vocale Dawwyd\\[0.5em]Dossier Fonctionnel}}

\author{Benoit HOUDAYER \\ \href{mailto:benoit.houdayer@utbm.fr}{benoit.houdayer@utbm.fr}
\and Anthony RUHIER \\ \href{mailto:anthony.ruhier@utbm.fr}{anthony.ruhier@utbm.fr}}

\date{21 janvier 2016}

\pagestyle{fancy}
\setlength{\headheight}{12pt}
\fancyhf{}
\fancyhead[L]{Benoit HOUDAYER, Anthony RUHIER}
\fancyhead[R]{Dawwyd -- Dossier Fonctionnel}
\geometry{headsep=5ex}

\def\ctabular{}


\begin{document}
    \maketitle
    \thispagestyle{empty}
    \tableofcontents
    \listoffigures

    \section*{Historique des modifications}

    \begin{table}
    \centering

    \begin{tabular}{|l|l|l|l|}
        \hline
        version & date & auteur & modification \\
        \hline
        0.1 & 21 janvier 2016 & BHD & création du document \\
        0.2 & 21 janvier 2016 & ARH & ajout du planing détaillé \\
        \hline
    \end{tabular}
    \end{table}

    \afterpage{\cfoot{\thepage}}
    \newpage

    \section{Fonctionnalités}
    Dawwyd peut être décrit comme étant composé de 3 fonctions principales :
    \begin{description}
        \item[la transcription] est la fonction permettant de tranformer la voix
            de l'utilisateur sous forme de texte, analysable par l'application.
        \item[l'analyse sémantique] est la compréhension du sens de la phrase
            de l'utilisateur menant à la prise de décision sur
            l'action à effectuer.
        \item[l'intégration avec les application de l'utilisateur] est constituée
            de l'ensemble des modules interagissant avec les applications,
            et fournissant chacun une ou plusieurs actions.
    \end{description}

    \subsection{Transcription}
    L'application doit être capable de transcrire
    les paroles de l'utilisateur pour pouvoir interpréter la
    commande
    \begin{itemize}
        \item 60\% de reconnaissance correcte
        \item Consommation mémoire réduite pour pouvoir cibler des appareils
            dans le domaine de l'embarqué.
    \end{itemize}

    \subsection{Analyse sémantique}
    Après transcription, une commande doit être interprétée pour déclencher une
    action.
    \begin{itemize}
        \item choix d'une action en 3 secondes ou moins
        \item Modulaire
    \end{itemize}

    \subsection{Intégration avec les applications de l'utilisateur}
    \paragraph{}
    Les actions à effectuer par Dawwyd auprès des applications de l'utilisateur
    sont matérialisée par des modules. De cette façon, il est possible
    d'intégrer plus facilement de nouvelles fonctionnalités.

    \paragraph{}
    Les modules ne sont pas critiques pour le fonctionnement de l'application
    mais enrichissent considérablement l'expérience de l'utilisateur. Pour
    cette raison, la quantité de modules implémentés dépendra de la rapidité
    de l'avancement du projet.

    \paragraph{}
    Au démarrage, Dawwyd devra découvrir le parc applicatif installé de façon à
    ne charger que les modules qui seront utilisables par l'utilisateur. Il est
    inutile de charger un module gérant une application qui n'est pas présente.


    \subsubsection{Lecteur de musique}
    L'intégration à pour but de couvrir les fonctionnalités pour une utilisation
    basique du lecteur; le but étant de permettre la navigation dans un album
    une fois la lecture lancée manuellement par l'utilisateur.

    \paragraph{}
    Ainsi, les fonctions implémentées seront :
    \begin{itemize}
        \item piste suivante/précédente
        \item lecture/pause
        \item augmenter/diminuer le volume
    \end{itemize}

    \paragraph{}
    Les contraintes associées sont alors :
    \begin{itemize}
        \item les probabilités de succès de la reconnaissance vocale devront
            tenir les 60\% annoncés plus haut, malgré le bruit ambiant dégagé
            par la musique.
    \end{itemize}

    \subsubsection{Navigateur internet}
    Cette intégration permettra de demander l'ouverture du navigateur sur une
    page précise.

    \begin{itemize}
        \item la page à ouvrir est précisée dans la commande par l'utilisateur
    \end{itemize}

    \paragraph{}
    Ce qui conduit à fixer les contraintes suivantes:
    \begin{itemize}
        \item l'url (l'adresse complète du site à atteindre) ne devra pas être
            dictée entièrement par l'utilisateur, mais fonctionnera par alias.
        \item ouvrir un onglet sur le navigateur par défaut
    \end{itemize}

    \subsubsection{Météo}
    \paragraph{}
    Dawwyd récupère les informations sur la météo et les énonce à
    l'utilisateur.

    \begin{itemize}
        \item Dawwyd récupère les informations sur la ville de résidence de
            l'utilisateur
        \item les informations sont énoncées de manière concise et
            compréhensible
    \end{itemize}

    \subsubsection{Actualités}
    \paragraph{}
    Dawwyd récupère les titres de l'actualité et les énonce à l'utilisateur.

    \begin{itemize}
        \item les titres sont lus dans l'ordre chronologique à l'utilisateur
        \item les informations sont énoncées de manière concise et
            compréhensible
    \end{itemize}


\end{document}
